\documentclass{article}
\usepackage[utf8]{inputenc}
\usepackage[brazil]{babel}

\title{\texttt{ecothesis} - Um modelo em \LaTeX~ para Trabalhos de Conclusão de Curso, Teses e Dissertações\\\textbf{Guia de utilização}}

\author{Vinícius Barros Rodrigues}

\newcommand{\modelo}{\texttt{\textbf{ecothesis}}}

\begin{document}

\maketitle

\section{Organograma dos arquivos}

Os arquivos do \modelo~ estão organizados da seguinte mandeira:


\begin{itemize}
\ttfamily
 \item ecothesis:
 \begin{itemize}
  \item dados.tex
  \item main.tex
  \item res\_abs.tex
  \item preambulo:
  \begin{itemize}
   \item agradecimentos.tex
   \item resumo.tex
   \item epigrafe.tex
   \item dedicatoria.tex
   \item capa.tex
   \item aprovacao.tex
   \item abstract.tex
  \end{itemize}
    \item capitulos:
    \begin{itemize}
     \item cap01:
     \begin{itemize}
      \item figura01.jpg
      \item textoCap01.tex
      \item referencias.bib
     \end{itemize}

     \item $[$...$]$
    \end{itemize}

 \end{itemize}


\end{itemize}

\section{Guia de utilização}

\subsection{Pacotes necessários}

Para a compilação do documento são necessários alguns pacotes. Tentamos utilizar os mais comuns e básicos possíveis. Também adicionamos alguns outros pacotes comumente utilizados em artigos e documentos científicos, como o para adicionar folhas no formato de paissagem e fórmulas matemáticas. Os pacotes sugeridos para a instalação são os seguintes (em ordem alfabética):

\begin{description}
 \item \texttt{array}, \texttt{amsmath}, \texttt{amssymb}, \texttt{booktabs}, \texttt{chapterbib}, \texttt{color}, \texttt{colortbl}, \texttt{fancyhdr}, \texttt{float}, \texttt{footmisc}, \texttt{framed}, \texttt{graphicx}, \texttt{hyperref}, \texttt{lineno}, \texttt{lipsum}, \texttt{longtable}, \texttt{lscape}, \texttt{multirow}, \texttt{natbib},\\ \texttt{pdfpages}, \texttt{setspace}, \texttt{standalone}, \texttt{textcomp}, \texttt{titlesec}, \texttt{tocloft}, \texttt{xcolor}, \texttt{xspace}
\end{description}



\subsection{Dados para documento}

A primeira parte a ser configurada é o arquivo \texttt{\textbf{dados (.tex)}}. Nesse arquivo, alguns comandos foram criados para serem utilizados nos demais arquivos que compõem o documento e necessitam ser preenchidos. São eles:

\begin{itemize}
 \item Seu nome completo \texttt{($\backslash$nome)}: \textit{p. ex.} Artur Bernardes
 \item Tipo de documento \texttt{($\backslash$tipo)}: \textit{p. ex.} Tese
 \item Programa de Pós-graduação \texttt{($\backslash$programa)}: \textit{p. ex.} Entomologia
 \item O nível do curso ou programa \texttt{($\backslash$curso)}: \textit{p. ex.} Programa de Pós-Graduação
 \item Título do documento \texttt{($\backslash$titulo)}: \textit{p. ex.} Um modelo em \LaTeX~ para TCCs, Teses e Dissertações
 \item Titulação \texttt{($\backslash$titulop)}: \textit{p. ex.} Doctor Scientiae
 \item Cidade \texttt{($\backslash$cidade)}: \textit{p. ex.} Viçosa
 \item Estado e País \texttt{($\backslash$estado)}: \textit{p. ex.} Minas Gerais - Brasil
 \item Mês \texttt{($\backslash$mes)}: \textit{p. ex.} Julho
 \item Ano \texttt{($\backslash$ano)}: \textit{p. ex.} 2017
 \item Data de aprovação \texttt{($\backslash$aprovacao)}: \textit{p. ex.} 22 de março de 1988
 \item Instituição de ensino: \texttt{($\backslash$instituicao)}: \textit{p. ex.} Univerisidade Federal de Viçosa
 \item Membros da banca:
 \begin{itemize}
  \item Membro 01 \texttt{($\backslash$membroa)}: \textit{p. ex.} Delfim Moreira
    \item Membro 02 \texttt{($\backslash$membrob)}: \textit{p. ex.} Epitácio Pessoa
      \item Membro 03 \texttt{($\backslash$membroc)}: \textit{p. ex.} Raul Soares de Moura
        \item Membro 04 \texttt{($\backslash$membrod)}: \textit{p. ex.} Washington Luís
          \item Membro 05 \texttt{($\backslash$membroe)}: \textit{p. ex.} Júlio Prestes
 \end{itemize}
\end{itemize}

 O arquivo \texttt{\textbf{dados (.tex)}} também possui alguns campos necessários para as páginas de Resumo e Abstract:
 
 \begin{itemize}
  \item Nome para a citação no resumo \texttt{($\backslash$nomecite)}: \textit{p. ex.} BERNARDES, Artur
   \item Titulação abreviada \texttt{($\backslash$titulacao)}: \textit{p. ex.} D. Sc.
   \item Nome do orientador \texttt{($\backslash$orientador)}: \textit{p. ex.} Nome
   \item Mês e ano da aprovação \texttt{($\backslash$aprovacaoPT)}: \textit{p. ex.} outubro, 2017
   \item Mês e ano da aprovação em inglês \texttt{($\backslash$aprovacaoEng)}: \textit{p. ex.} October, 2017
   \item Título do trabalho em inglês \texttt{($\backslash$tituloEng)}: \textit{p. ex.} Template in \LaTeX~ for Thesis and Dissertation
 \end{itemize}


\subsection{Preâmbulo}

Em seguida deverá ser preenchido o que chamamos de preâmbulo do documento. Essa parte constitui as seções que precedem os capítulos. Os arquivos necessários estão separados na pasta ``preambulo''. Nesta pasta, apenas alguns arquivos podem ser editados \textbf{nos locais indicados}:

\begin{itemize}
 \item \texttt{\textbf{capa.tex}}: Capa do documento. Não é necessário editar.
 \item \texttt{\textbf{aprovacao.tex}}: Folha de aprovação do documento onde os membros da banca assinam. Não é necessário editar.
 \item \texttt{\textbf{dedicatoria.tex}}: Folha opcional. Editar no local indicado.
 \item \texttt{\textbf{epigrafe.tex}}: Folha opcional. Editar no local indicado.
 \item \texttt{\textbf{agradecimentos.tex}}: Folha opcional. Editar no local indicado.
\end{itemize}

Nenhum desses arquivos deverá ser compilado. Eles serão utilizados pelo arquivo principal: \texttt{\textbf{main (.tex)}}. Os arquivos \texttt{\textbf{dedicatoria (.tex)}}, \texttt{\textbf{epígrafe (.tex)}} e \texttt{\textbf{agradecimentos (.tex)}} são opcionais e a remoção de um desses arquivos deve ser feita no arquivo \texttt{\textbf{main (.tex)}}.

Na pasta ``preambulo'' também estão disponíveis os arquivos:

\begin{itemize}
 \item \texttt{\textbf{resumo.tex}}: Irá gerar a página do Resumo.
 \item \texttt{\textbf{abstract.tex}}: Irá gerar a página do Asbtract.
\end{itemize}

Ambos os arquivos \texttt{\textbf{resumo (.tex)}} e \texttt{\textbf{abstract (.tex)}} na pasta ``preambulo'' não necessitam ser editados.

\subsection{Resumo e abstract}

O Resumo e o Asbtract do documento devem ser colocados no arquivo \texttt{res\_abs (.tex)} nos locais indicados. Os títulos desses textos serão gerados automaticamente com base nas informações adicionadas no arquivo \texttt{dados (.tex)}.

\subsection{Capítulos}

Os capítulos devem ser colocados separadamente na pasta ``capitulos'', juntamente com o banco de dados de refências \texttt{(.bib)} e as figuras utilizadas.

\subsubsection{Referências}

Nós utilizamos o pacote \texttt{chapterbib}, para separar as referências por capítulos. Essa opção está como padrão no \modelo~ e \textbf{é necessário compilar separadamente os capítulos} para que funcione devidamente. Isso é necessário para gerar um arquivo \texttt{.bbl} por capítulo, que por sua vez será utilizado pelo arquivo \texttt{\textbf{main.tex}}.


\subsection{Compilação}

Para gerar o documento final, basta compilar o arquivo \texttt{\textbf{main.tex}} utilizando o \texttt{PDFLaTeX}. Recomendamos gerar inicialmente as referências através do \texttt{Bibtex} e, em seguida, os índices através do \texttt{MakeIndex}.

\end{document}
